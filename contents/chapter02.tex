% !TeX root = ../main.tex
\chapter{Literature Review}
\section{Residential Shifts and Internal Migration}\label{sec:2_residential_shift_and_internal_migration}
% CDRs is a collection of geotagged phone call records (Table~\ref{tab:example_cdr}) over a period of time.
% Although the observation units are phone numbers rather than actual phone users, we will proceed to use phone users as the sample units since, on average, people have 1.2 phone devices.
% By looking at individual level, we can trace phone users' geographical appearances with temporal dimension.
% Based on the spatial patterns distributed on the map, we can estimate their significant locations, such as their residential coordinates and workplace location.
% Besides, the temporal dimension offers a particularly exciting opportunity: we can identify events of residential shifts, depicting migration flows that were previously impossible to capture at such scale and precision through traditional census-based survey.


% <elaborating the importance of the topic of internal migration>
Internal migration signifies population flow that occurs within a country, which is a phenomenon that has captured economists' attention for over a century.
This strand of literature often involves modeling migration decisions (\cite{hunt2004north}, \cite{espindola2006harris}, \cite{wang2023job}) or inspecting the impacts of migration on the destination region (\cite{boustan2010effect}, \cite{bryan2019aggregate}, \cite{imbert2022migrants}).

% <explaining our contribution> <note that what we have done differently will be elaborated in other section>
As phone call records are geotagged, and we assume individuals spend most of their time at home during nighttime, we can refer to the geographic location of the most representative telecom base station as an individual's home coordinate.
Consequently, we can identify those who change their residential locations, depicting internal migration flows.
Although the most typical internal migration is rural-urban migration for reasons such as finding better jobs or seeking education opportunities, we don't focus on any particular context and instead concentrate on detecting residential shifts and their influences on human behaviors.

% One of our methodological contributions is identifying internal migration flows through novel data sources, which were previously impossible to capture at such scale and precision through traditional census-based surveys.

% Specifically, we employ CDRs and develop a systematic pipeline with two stages to identify individuals changing their residential locations.

Utilizing CDRs to identify internal migration flow has several advantages. First, CDRs capture mobility patterns for virtually all mobile phone users in a region, including populations often underrepresented in conventional surveys such as transient residents, undocumented individuals, and those reluctant to participate in formal governmental data collection. Second, CDRs provide continuous temporal coverage rather than the periodic snapshots offered by censuses, enabling the detection of short-term or seasonal relocations. Third, this approach is also cost-effective compared to the large amount of money and human resources devoted to completing a census, as telecommunication companies automatically collect these phone records for billing purposes.

% \section{Mobility and Mobile Communication Features}
% Whether immigrants are integrated into the destination region is a topic of great interest to policymakers. \cite{voukelatou2021measuring} provides an overview on utilizing CDRs to measure objective well-being on various dimensions, such as job opportunities, socioeconomic development and so on by analyzing mobility and mobile communication behaviors. Therefore, these two groups of features could be used to measure the integration of immigrants into the destination region. To clarify, what we mean by integration is that immigrants going back to the lifestyle before the residential shift, getting used to the new environment. Moving to a new environment, people might change their communication behaviors toward their old friends or their mobility patterns might be substantially different from their old lifestyle. We have confirmed the changes after immigration and found short-term effects, which means they instantly find their own way to adapt to the new environment.


% One of the mobile communication network features, the contact distance, captures the average geographical distance between individuals and their contacts, offering insight into the spatial reach of their social interactions.
% A higher contact distance suggests that a person frequently interacts with others who live far away, which may indicate broader social ties, long-distance relationships, or mobility patterns such as migration or commuting, and as mentioned previously, it shows a significant change after a residential shift.
% Therefore, we reckon that the contact distance is a good indicator of community integration, providing useful policy implications.

\section{Home Location Estimation through CDRs}\label{sec:2_home_location_estimation_through_cdrs}
% A critical preliminary step for identifying residential shifts is estimating phone users' home coordinates.
% It is important to clarify that the coordinates attached to each phone call record don't precisely represent the exact geographical position of either the caller or callee.
% Rather, they are the coordinates of the telecom base station handling the call, which serve as a spatial approximation for the phone user's position while initiating or receiving phone calls.
% The approximation is not always accurate due to the load balancing mechanism.

Most of the studies estimate home locations through CDRs by selecting the locations of the telecom base station that handles the call events most frequently over the whole sample period (\cite{cho2011friendship}, \cite{phithakkitnukoon2012socio}), weekly (\cite{referral_effect_2023aer}), or monthly (\cite{phithakkitnukoon2022inferring}) from the nighttime call records.
This simple approach seems to be acceptable for people who have a large amount of phone call records. However, for those who have limited observations of call events, the simple approach is not reliable.

A more robust approach would be running a spatial clustering algorithm over a set of telecom base stations' locations (\cite{isaacman2011identifying}, \cite{yang2014identifying}, \cite{ayesha2019user}).
Our estimation strategy of residential location encompasses two stages, and the first stage recognizes the clustered patterns and leverage DBSCAN (\cite{ester1996density}), a renowned machine learning algorithm in the clustering domain, to uncover them. Note that \cite{ayesha2019user} also leveraged DBSCAN to estimate home locations.
% For the angle of clustering, what the clustering's goal is to recognize the neighbors of each point.
% The upper plot of Figure \ref{fig:cluster} visualizes the spatial clustering results where there are four clusters identified and colored by the cluster index.


DBSCAN's flexibility has made it a popular choice for analyzing spatial patterns (\cite{yang2014identifying}, \cite{shi2014density}, \cite{dominguez2017sensing}) and mobile communication behaviors (\cite{karahoca2006comparing}, \cite{jabbar2020fraud}).
We inherit this idea, including it in our two-staged approach, which carries the specific goal of identifying residential shifts.
Our approach makes this identification feasible by incorporating temporal information following the spatial clustering process.
Even more recent developments in significant location inference (\cite{tongsinoot2017exploring}, \cite{luo2020research}) don't explicitly consider the situation where people might change their home locations.


% In the second stage, we implement an extremely simple temporal filtering technique to remove unrepresentative clusters possibly caused by temporary visits or anomalous cases. The remaining clusters are well-representative home clusters, the centers of which are identified as home locations.


\section{Detection of Residential Shifts through CDRs}\label{sec:2_detection_of_residential_shifts_through_cdrs}
% Several studies have already taken advantage of CDRs to identify residential relocation.
Most of the works on home location inference are not designed for detecting home location shifts in that they often estimate one location over the whole sample period.
However, we can still apply these approaches on multiple fixed-size time window (\cite{blumenstock2012inferring}, \cite{phithakkitnukoon2022inferring}, \cite{blumenstock2025migration}), e.g., daily, weekly, or monthly, or between two time periods (\cite{lai2019exploring}, \cite{dias2022framework}) and if more than one home location is found, we can consider it as a residential shift.
\cite{phithakkitnukoon2022inferring} is the case where they apply the simple approach on each month to detect residential shifts while \cite{dias2022framework} adopt \cite{isaacman2011identifying}'s home location estimation method on January to March 2013 and July to September 2013, respectively.
This strategy for inferring residential shift requires several predefined parameters, such as the minimum time span of each residential location to exclude short-term visits or distance threshold to define the separation of two home locations.
Hence, a heavy procedure of sensitive analysis is required to select the appropriate parameters.

\cite{buchel2020calling} also utilize CDRs to identify residential shift and benefits from high-quality data billing addresses, making residential estimation extremely precise and identification of migrants is simply based on whether individuals change their residential locations.
However, as such data is not available in most scenarios, thereby failing to be widely applicable.

\cite{chi2020general} is a closely related work.
They abandon the first stage by clustering coordinates of telecom base stations if they are in the same administrative district (e.g., prefectures in China).
Furthermore, they apply the clustering algorithm on time axis for each district to identify contiguous time segments and further merge segments tailored to a particular district if no other segments from the other district are found in the same time window defined by the target merging segments.
At the last step, they allow for overlap between segments from different districts to form home locations at different periods.

Our two-stage approach provides fine spatial resolution through first-stage spatial clustering, though this isn't strictly necessary for cross-district migration flows.
The second stage uses temporal filtering---simpler than temporal clustering---that identifies home clusters by assuming the largest nighttime cluster is most likely to be one of the "home" clusters as people won't change their home locations too often.
However, their second stage may be more robust as it allows for overlap between clusters.

Our approach stands out for being trivial and efficient.
It relies solely on CDRs, requiring no additional information.
There is only one parameter: the maximum distance between two locations to be considered neighbors, and since we are interested in cross-district migration flows, sensitivity analysis of this parameter is not strictly necessary.
Furthermore, we maximally exploit the full range of temporal information.
