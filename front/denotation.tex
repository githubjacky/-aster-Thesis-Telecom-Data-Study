% !TeX root = ../main.tex

\begin{denotation}[3cm]

\item[HPC]{
  高性能計算 (High Performance Computing)
}

\item[cluster]{
  集群
}

\item[Itanium]{
  安騰
}

\item[SMP]{
  對稱多處理
}

\item[API]{
  應用程序編程接口
}

\item[PI]{
  聚酰亞胺
}

\item[MPI]{
  聚酰亞胺模型化合物,N-苯基鄰苯酰亞胺
}

\item[PBI]{
  聚苯並咪唑
}

\item[MPBI]{
  聚苯並咪唑模型化合物,N-苯基苯並咪唑
}

\item[PY]{
  聚吡嚨
}

\item[PMDA-BDA]{
  均苯四酸二酐與聯苯四胺合成的聚吡嚨薄膜
}

\item[$\Delta G$]{
  活化自由能 (Activation Free Energy)
}

\item[$\chi$]{
  傳輸系數 (Transmission Coefficient)
}

\item[$E$]{
  能量
}

\item[$m$]{
  質量
}

\item[$c$]{
  光速
}

\item[$P$]{
  概率
}

\item[$T$]{
  時間
}

\item[$v$]{
  速度
}

\item[勸學]{
  君子曰:學不可以已。青,取之於藍,而青於藍;冰,水為之,而寒於水。木直中繩。輮以為輪,其曲中規。雖有槁暴,不覆挺者,輮使之然也。故木受繩則直,金就礪則利,君子博學而日參省乎己,則知明而行無過矣。吾嘗終日而思矣,不如須臾之所學也;吾嘗跂而望矣,不如登高之博見也。登高而招,臂非加長也,而見者遠;順風而呼,聲非加疾也,而聞者彰。假輿馬者,非利足也,而致千裏;假舟楫者,非能水也,而絕江河,君子生非異也,善假於物也。積土成山,風雨興焉;積水成淵,蛟龍生焉;積善成德,而神明自得,聖心備焉。故不積跬步,無以至千裏;不積小流,無以成江海。騏驥一躍,不能十步;駑馬十駕,功在不舍。鍥而舍之,朽木不折;鍥而不舍,金石可鏤。蚓無爪牙之利,筋骨之強,上食埃土,下飲黃泉,用心一也。蟹六跪而二螯,非蛇鱔之穴無可寄托者,用心躁也。—— 荀況
}

\end{denotation}
