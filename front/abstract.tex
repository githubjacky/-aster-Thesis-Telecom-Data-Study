% !TeX root = ../main.tex

\begin{abstract}
我們使用大量匿名通話記錄,用以研究居住地變遷與智慧型手機採用對移動與通訊行為的影響。資料涵蓋每月超過十五億通話,涉及超過五十萬個電話號碼,而時間涵蓋2013年8月至2014年5月。

我們發現居住地搬遷具有顯著的時間變異效應。遷徙者在搬遷期間傾向於更頻繁地通話,建立更多元的聯絡關係,且主要與原本身處遠距的朋友互動。然而,這些效應會迅速消退回原本水準,或持續發展為負向趨勢,例如互動對象變得較不多元,或聯絡距離縮短。從移動行為的角度來看,搬遷會導致使用者的活動範圍擴大,並出現較難預測的移動模式,儘管這些效應隨時間也會逐漸趨於穩定並變得可預測。

在採用智慧型手機後,移動模式出現明顯(近乎靜態)的上升變化,可能是因為科技在陌生環境中提供協助。例如,移動的不確定性上升,同時出現較明確的方向偏好。

本研究顯示,針對此行為變化,在大規模遷移或手機科技升級的範疇下,政策應更加關注行動與交通建設的需求。
\end{abstract}


\begin{abstract*}
\setstretch{1.6}
We use over 1.5 billion anonymized call records per month spanning from August 2013 to May 2014, where more than 500,000 phone numbers are involved, to study the impacts of residential shifts (events of changing home locations) and smartphone adoption on mobility and communication behaviors.

We find significant time-variant effects for residential relocations. Migrants tend to call more frequently, engage in more diverse contact relationships, and primarily interact with existing distant friends during relocation periods. These effects quickly fade to original levels or continuously evolve toward negative states, such as less diverse interactions or shorter contact distances. From a mobility perspective, residential relocations cause users to have larger exploration areas and highly unpredictable movement patterns, though these effects also shift to more predictable movement over time.

The notable upward shifts (nearly static) in mobility patterns after smartphone adoption are likely due to technological assistance in unfamiliar environments. For example, movement unpredictability increases along with relatively clearer directional preferences.

Our work provides evidence-based policy implications that mobile and transportation infrastructure needs are worth considering during periods when large-scale population displacements or mobile technology upgrades occur.
\end{abstract*}
