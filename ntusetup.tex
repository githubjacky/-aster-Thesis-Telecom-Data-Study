% !TeX root = ./main.tex

% --------------------------------------------------
% 資訊設定(Information Configs)
% --------------------------------------------------

\ntusetup{
  university*   = {National Taiwan University},
  university    = {國立臺灣大學},
  college       = {社會科學院},
  college*      = {College of Social Sciences},
  institute     = {經濟學系},
  institute*    = {Department of Economics}, % 請確認您的科系為Department of xxx或Institute of xxx
  title         = {電信數據之時空分析:居住地遷移與智慧型手機使用對移動行為與通訊模式之影響},
  title*        = {Spatio-temporal Analysis of Telecommunications Data: Effects of Residential Shifts and Smartphone Adoption on Mobility Patterns and Communication Behavior},
  author        = {葉秀軒},
  author*       = {Hsiu-Hsuan Yeh},
  ID            = {R12323011},
  advisor       = {謝志昇},
  advisor*      = {Chih-Sheng Hsieh},
  date          = {2025-07-31},         % 若註解掉,則預設為當天
  oral-date     = {2025-07-25},         % 若註解掉,則預設為當天
  DOI           = {10.6342/NTU202502893},
  keywords      = {時空分析, 電信網路, 居住遷移, 智能手機採用, 移動行為, 通訊模式},
  keywords*     = {Spatial-Temporal Analysis, Telecommunications, Residential shifts, Smartphone adoption, Mobility Patterns, Communication Behavior},
}

% --------------------------------------------------
% 加載套件(Include Packages)
% --------------------------------------------------

\usepackage[sort&compress]{natbib}      % 參考文獻
\usepackage{amsmath, amsthm, amssymb}   % 數學環境
% \usepackage{ulem, CJKulem}              % 下劃線、雙下劃線與波浪紋效果
\usepackage[normalem]{ulem}
\usepackage{booktabs}                   % 改善表格設置
\usepackage{multirow}                   % 合併儲存格
\usepackage{diagbox}                    % 插入表格反斜線
\usepackage{array}                      % 調整表格高度
\usepackage{longtable}                  % 支援跨頁長表格
\usepackage{paralist}                   % 列表環境


\usepackage{lipsum}                     % 英文亂字
\usepackage{zhlipsum}                   % 中文亂字

\usepackage{algorithm}
\usepackage{algorithmic}
\usepackage{mathrsfs}
\usepackage{bbm}
\usepackage{hyperref}
\usepackage{tabularray}
\usepackage{siunitx}
\usepackage{svg}

% --------------------------------------------------
% 套件設定(Packages Settings)
% --------------------------------------------------
\theoremstyle{definition}
\newtheorem{definition}{Definition}[section]

\theoremstyle{definition}
\newtheorem{function}{Function}[section]

\newtheorem{theorem}{Theorem}[section]
\newtheorem{lemma}[theorem]{Lemma}

\svgsetup{inkscapelatex=false}
